\documentclass[12pt,fleqn]{jsarticle}
\usepackage[top=10mm,bottom=15mm,left=18mm,right=18mm]{geometry}
\usepackage{listings}
\usepackage{empheq}
\usepackage[dvipdfmx]{color}
\usepackage{ascmac}
\renewcommand{\labelenumi}{(\theenumi)}  % enumerateの番号のスタイル

%%%%% lstlistingのスタイル定義 ここから
\lstdefinestyle{ex}{
 frame={trbl},
 frameround={tttt},
 lineskip=-0.7ex,
 numbers=none
}

\lstset{%default
basicstyle={\ttfamily},
 breaklines=true,
 escapechar=\`,
 columns=[l]{fullflexible},
 keepspaces=true,
 showstringspaces=false,
 xrightmargin=0em,
 xleftmargin=1em,
 stepnumber=1,
 lineskip=-0.7ex,
 frame={lrtb},
 frameround={nnnn},
 numberstyle={\scriptsize},
 numbers=left
}
%%%%% lstlistingのスタイル定義 ここまで

\begin{document}

\section*{HI4実験 SMTソルバ演習レポート}

\begin{flushright}
 提出日: 20YY.MM.DD \\
 HI4 NN番 氏名
\end{flushright}

{\color{blue}
\begin{boxnote}
レポートの次のような構成で記述する.
\begin{enumerate}
 \item 表題,提出日,氏名
 \item 各課題の解答
 \begin{enumerate}
 \item 課題名と課題内容の要約
 \item 問題への解答(説明)
 \item プログラム
 \item 実行例(入力によって実行結果が変わる場合は,複数の実行例を記述する.)
 \end{enumerate}
\item 感想
\end{enumerate}

\end{boxnote}
}

\section*{\underline{課題smt-2: 解のない連立方程式}}
解がないような連立方程式を考え,その解をZ3に求めさせようとした場合に,
UNSAT(充足不能,解なし)となることを確認するプログラムを作成せよ.

\subsection*{解答}

次のような連立方程式を考えた.

\begin{eqnarray*}
  4x  +  2y & =  2   \\
    :
\end{eqnarray*}

この連立方程式は..


\subsection*{プログラム}
\begin{lstlisting}
from z3 import *
    :
\end{lstlisting}

\subsection*{実行例}
\subsubsection*{実行例1}

(入力によって実行結果が変わる場合は,複数の実行例を記述する.)

\begin{lstlisting}[style=ex]
$ python3 eq.py
     :
\end{lstlisting}

\subsubsection*{実行例2}

\section*{\underline{課題smt-3: ...}}


\section*{感想}




\end{document}
